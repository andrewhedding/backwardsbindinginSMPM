\documentclass[12pt]{article}
\usepackage[margin=1in]{geometry}



\usepackage{linguex}

\usepackage[normalem]{ulem}

\usepackage{pstricks}
\usepackage{pst-jtree}
\usepackage{pst-xkey}
\usepackage{tipa}

\usepackage[round]{natbib}
\setlength{\bibsep}{3pt plus 0.5ex}
\bibliographystyle{apa}
\title{TBD*\footnotetext{*Authors are listed in alphabetical order. All remaining errors are each other's.}}
\author{Ben Eischens \& Andrew Hedding}
\date{\today}
\begin{document}
\maketitle

\section{Introduction}

San Mart\' in Peras Mixtec has a canonical VSO word order, as shown below. A post-verbal pronoun or R-expression specifies a third person subject: 

\exg. shi\textsuperscript{3}shi\textsuperscript{2} Pe\textsuperscript{3}bro\textsuperscript{2} shi\textsuperscript{1}ta\textsuperscript{1} \\
eat.\textsc{cont} Pedro tortilla \\
``Pedro eats tortillas"

\exg. shi\textsuperscript{3}shi\textsuperscript{2} ra\textsuperscript{1} shi\textsuperscript{1}ta\textsuperscript{1} \\
eat.\textsc{cont} \textsc{3.sg.masc} tortilla \\
``He eats tortillas"

For first and second person subjects, an agreement clitic fuses with the verbal stem:

\exg. shi\textsuperscript{3}sh-i\textsuperscript{21} shi\textsuperscript{1}ta\textsuperscript{1} \\
hit.\textsc{compl}-1\textsc{sg} Pedro \\
``I eat tortillas"

\exg. shi\textsuperscript{3}sh-\~u\textsuperscript{2}  shi\textsuperscript{1}ta\textsuperscript{1}\\
hit.\textsc{compl}-2\textsc{sg} tortilla \\
``You eat tortillas"

Embedded clauses generally follow the pattern below, which is \textsubscript{Matrix}[V S (O) \textsubscript{Embedded}[(Comp) V S (O)]]: \\

\exg. [ k\~a\textsuperscript{3}'\~a\textsuperscript{2} \~na\textsuperscript{3} [ ke\textsuperscript{3}ba\textsuperscript{2}'a\textsuperscript{2} \~na\textsuperscript{3} ] ] \\
[ think.\textsc{cont} 3\textsc{sg.fem} [ win.\textsc{cont} 3\textsc{sg.fem} ] ] \\
``She thinks that she is winning" (from Ostrove 2018)

In general, the matrix clause must have a subject \Next, and this pattern is followed in most control constructions. Leaving the matrix verb without an overt subject is usually disallowed \NNext. In addition, there must be an overt subject in the embedded clause; null \textsc{pro} is not allowed (8).

\exg. ko\textsuperscript{32}ni\textsuperscript{1} Pe\textsuperscript{3}bro\textsuperscript{2} ku\textsuperscript{2}shi\textsuperscript{2} ra\textsuperscript{2} shi\textsuperscript{1}ta\textsuperscript{1} \\
want.\textsc{cont} Pedro eat.\textsc{pot} \textsc{3.sg.masc} tortilla \\
``Pedro wants to eat tortillas"

\exg. *ko\textsuperscript{32}ni\textsuperscript{1} ku\textsuperscript{2}shi\textsuperscript{2} Pe\textsuperscript{3}bro\textsuperscript{2} shi\textsuperscript{1}ta\textsuperscript{1} \\
want.\textsc{cont} eat.\textsc{pot} Pedro tortilla \\
Intended: ``Pedro wants to eat tortillas"

\exg. *ko\textsuperscript{32}ni\textsuperscript{1} Pe\textsuperscript{3}bro\textsuperscript{2} ku\textsuperscript{2}shi\textsuperscript{2} shi\textsuperscript{1}ta\textsuperscript{1} \\
want.\textsc{cont} Pedro eat.\textsc{pot} tortilla \\
Intended: ``Pedro wants to eat tortillas"


However, there are certain constructions in SMPM that allow this exact pattern. Interestingly, the verb meaning `to start' permits a construction in which the matrix clause does not contain a subject:

\exg. ki\textsuperscript{1}sha\textsuperscript{2} ka\textsuperscript{3}'\~a\textsuperscript{2} ra\textsuperscript{1} shi\textsuperscript{3}'\~i\textsuperscript{2} \~na\textsuperscript{3} \\
begin.\textsc{compl} talk.\textsc{cont} \textsc{3.sg.masc} with \textsc{3.sg.fem}\\
``He began to talk to her"

As expected, a construction in which there is an overt subject in the matrix clause and a silent PRO in the embedded clause, is disallowed: 

\exg. *ki\textsuperscript{1}sha\textsuperscript{2} ra\textsuperscript{1} ka\textsuperscript{3}'\~a\textsuperscript{2} shi\textsuperscript{3}'\~i\textsuperscript{2} \~na\textsuperscript{3} \\
begin\textsc{compl} \textsc{3.sg.masc} talk.\textsc{cont} with \textsc{3.sg.fem} \\
Intended: ``He began to speak with her"

What is surprising about this pattern of data is that, in general, SMPM disallows null subjects like \textsc{pro}. In fact, it is not possible for a locally bound \textsc{pro} to serve as the subject of an embedded clause (8). However, it \emph{is} possible for there to be a gap in matrix subject position in constructions with the verb `begin.' This is the opposite pattern of control observed in languages like English, where \textsc{PRO} must be bound by a matrix subject, as in \Next below, where a dashed line represents binding:

\ex. \textsubscript{\textsc{cp}}[ \rnode{A}I\textsubscript{k} want \textsubscript{\textsc{cp}}[ \textsc{\rnode{B}pro}\textsubscript{k} to eat tortillas ]]
\ncbar[angleA=270,linestyle=dashed]{->}{A}{B}

\section{Language Background}

\section{Polinsky and Potsdam}

\section{Application to Mixtec}

\section{Discussion and Implications}

\section{Conclusion}


\nocite{polinskandpotsdam2002}
\nocite{ostrove2018}
\footnotesize{\bibliography{Controlbib}}	
\end{document}